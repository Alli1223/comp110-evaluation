% Please do not change the document class
\documentclass{scrartcl}

% Please do not change these packages
\usepackage[hidelinks]{hyperref}
\usepackage[none]{hyphenat}
\usepackage{setspace}
\doublespace

% You may add additional packages here
\usepackage{amsmath}
\usepackage{graphicx}
\usepackage{wrapfig}
\graphicspath{ {./images/} }

% Please include a clear, concise, and descriptive title
\title{Evaluation} 

% Please do not change the subtitle
\subtitle{COMP150 - Evaluation}

% Please put your student number in the author field
\author{1507516}

\begin{document}

\maketitle

\abstract{%Abstract }

\section{Introduction}

This is a reflective report that proposes three weaknesses that occurred during the second semester projects, then suggests how these issues could be overcome for future projects. My three weakest key skills are Communication in group projects, Discipline and Requesting support from peers and tutors. These are all skills that need to be improved for second year.

\section{Weakest Key Skills}

\subsection{Weakness One: Communication in group projects}

%\subsection{Description} 
In semester two we had a large group project that was to make a desktop game with a group of four. This was the first time I have been working on a game collaboratively and I felt that I did not communicate with my team often enough, which impacted the amount and quality of the work that was produced.

%\subsection{Interpretation & Reflection} 
Communication is vital to being an effective team in group projects, however in our group project there was almost no communication for the first 2 months of the project, and only in the last week of the project did we start communicating.

I feel this was mainly due to me and our group not following the scrum methodology enough.

%\subsection{Outcome(SMART)} 

To overcome this weakness, I am planning a summer games project with some friends back home, in which I will attempt to follow scrum closely and do daily stand up meetings.

\subsection{Weakness Two: Discipline}

%\subsection{Description} 
Often I find it hard to motivate myself to work, especially when the task at hand can seem very daunting. I will tend to put it off and work on something I enjoy, instead of getting frustrated with something I don't understand.

%\subsection{Interpretation & Reflection} 

For example, I struggled with trying to get std::vector to store values of the neighboring cells in the group game project, thus this demotivated me a lot to work on the group project as I felt I was unable to do it. 

%\subsection{Outcome(SMART)}
 
To overcome this, I aim to complete the book Design Patterns by Vlissides, John, et al before August furthermore I will also try to implement some of the more complex design patterns into our comp150 game project, as to become more familiar with OOP.

\subsection{Weakness Three: Requesting Support from Peers and Tutors}

%\subsection{Description} 

In one of my projects, I was stuck on one element of the game for a while, and was determined to try and do it myself. However I feel this is a bad approach to take as it wastes a lot of time when someone else can help you understand the problem much quicker.

%\subsection{Interpretation & Reflection} 

This could of easily been avoided if I asked my team or my tutors for help with this problem early on, instead of spending a few weeks trying to work it out myself.


%\subsection{Outcome} 
To try and resolve this problem in the future, over the course of the summer and the first semester next year I will attempt to communicate my issues when I know I am stuck on a problem. Furthermore this will hopefully be resolved when I follow scrum more closely and communicate my issues with the scrum master.

\section{Conclusion}

In conclusion, over the summer I will attempt to improve my communication and requesting support by working a group game project that will follow agile closely, and Discipline by the continuation of features to the comp150 game project that will make more effective use of OOP.




\end{document}
